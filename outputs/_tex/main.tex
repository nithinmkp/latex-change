% Options for packages loaded elsewhere
% Options for packages loaded elsewhere
\PassOptionsToPackage{unicode}{hyperref}
\PassOptionsToPackage{hyphens}{url}
%
\documentclass[
  12pt,
  a4paper,
  abstract]{scrartcl}
\usepackage{xcolor}
\usepackage{amsmath,amssymb}
\setcounter{secnumdepth}{5}
\usepackage{iftex}
\ifPDFTeX
  \usepackage[T1]{fontenc}
  \usepackage[utf8]{inputenc}
  \usepackage{textcomp} % provide euro and other symbols
\else % if luatex or xetex
  \usepackage{unicode-math} % this also loads fontspec
  \defaultfontfeatures{Scale=MatchLowercase}
  \defaultfontfeatures[\rmfamily]{Ligatures=TeX,Scale=1}
\fi
\usepackage{lmodern}
\ifPDFTeX\else
  % xetex/luatex font selection
\fi
% Use upquote if available, for straight quotes in verbatim environments
\IfFileExists{upquote.sty}{\usepackage{upquote}}{}
\IfFileExists{microtype.sty}{% use microtype if available
  \usepackage[]{microtype}
  \UseMicrotypeSet[protrusion]{basicmath} % disable protrusion for tt fonts
}{}
\makeatletter
\@ifundefined{KOMAClassName}{% if non-KOMA class
  \IfFileExists{parskip.sty}{%
    \usepackage{parskip}
  }{% else
    \setlength{\parindent}{0pt}
    \setlength{\parskip}{6pt plus 2pt minus 1pt}}
}{% if KOMA class
  \KOMAoptions{parskip=half}}
\makeatother
% Make \paragraph and \subparagraph free-standing
\makeatletter
\ifx\paragraph\undefined\else
  \let\oldparagraph\paragraph
  \renewcommand{\paragraph}{
    \@ifstar
      \xxxParagraphStar
      \xxxParagraphNoStar
  }
  \newcommand{\xxxParagraphStar}[1]{\oldparagraph*{#1}\mbox{}}
  \newcommand{\xxxParagraphNoStar}[1]{\oldparagraph{#1}\mbox{}}
\fi
\ifx\subparagraph\undefined\else
  \let\oldsubparagraph\subparagraph
  \renewcommand{\subparagraph}{
    \@ifstar
      \xxxSubParagraphStar
      \xxxSubParagraphNoStar
  }
  \newcommand{\xxxSubParagraphStar}[1]{\oldsubparagraph*{#1}\mbox{}}
  \newcommand{\xxxSubParagraphNoStar}[1]{\oldsubparagraph{#1}\mbox{}}
\fi
\makeatother


\usepackage{longtable,booktabs,array}
\usepackage{calc} % for calculating minipage widths
% Correct order of tables after \paragraph or \subparagraph
\usepackage{etoolbox}
\makeatletter
\patchcmd\longtable{\par}{\if@noskipsec\mbox{}\fi\par}{}{}
\makeatother
% Allow footnotes in longtable head/foot
\IfFileExists{footnotehyper.sty}{\usepackage{footnotehyper}}{\usepackage{footnote}}
\makesavenoteenv{longtable}
\usepackage{graphicx}
\makeatletter
\newsavebox\pandoc@box
\newcommand*\pandocbounded[1]{% scales image to fit in text height/width
  \sbox\pandoc@box{#1}%
  \Gscale@div\@tempa{\textheight}{\dimexpr\ht\pandoc@box+\dp\pandoc@box\relax}%
  \Gscale@div\@tempb{\linewidth}{\wd\pandoc@box}%
  \ifdim\@tempb\p@<\@tempa\p@\let\@tempa\@tempb\fi% select the smaller of both
  \ifdim\@tempa\p@<\p@\scalebox{\@tempa}{\usebox\pandoc@box}%
  \else\usebox{\pandoc@box}%
  \fi%
}
% Set default figure placement to htbp
\def\fps@figure{htbp}
\makeatother


% definitions for citeproc citations
\NewDocumentCommand\citeproctext{}{}
\NewDocumentCommand\citeproc{mm}{%
  \begingroup\def\citeproctext{#2}\cite{#1}\endgroup}
\makeatletter
 % allow citations to break across lines
 \let\@cite@ofmt\@firstofone
 % avoid brackets around text for \cite:
 \def\@biblabel#1{}
 \def\@cite#1#2{{#1\if@tempswa , #2\fi}}
\makeatother
\newlength{\cslhangindent}
\setlength{\cslhangindent}{1.5em}
\newlength{\csllabelwidth}
\setlength{\csllabelwidth}{3em}
\newenvironment{CSLReferences}[2] % #1 hanging-indent, #2 entry-spacing
 {\begin{list}{}{%
  \setlength{\itemindent}{0pt}
  \setlength{\leftmargin}{0pt}
  \setlength{\parsep}{0pt}
  % turn on hanging indent if param 1 is 1
  \ifodd #1
   \setlength{\leftmargin}{\cslhangindent}
   \setlength{\itemindent}{-1\cslhangindent}
  \fi
  % set entry spacing
  \setlength{\itemsep}{#2\baselineskip}}}
 {\end{list}}
\usepackage{calc}
\newcommand{\CSLBlock}[1]{\hfill\break\parbox[t]{\linewidth}{\strut\ignorespaces#1\strut}}
\newcommand{\CSLLeftMargin}[1]{\parbox[t]{\csllabelwidth}{\strut#1\strut}}
\newcommand{\CSLRightInline}[1]{\parbox[t]{\linewidth - \csllabelwidth}{\strut#1\strut}}
\newcommand{\CSLIndent}[1]{\hspace{\cslhangindent}#1}



\setlength{\emergencystretch}{3em} % prevent overfull lines

\providecommand{\tightlist}{%
  \setlength{\itemsep}{0pt}\setlength{\parskip}{0pt}}



 


% Packages %---------------
\usepackage{hyperref}
\usepackage[noabbrev,nameinlink,capitalise]{cleveref}
\usepackage{float} 
\usepackage{xpatch} 
\usepackage{rotating}
%\usepackage{adjustbox}
\usepackage{multirow}
\usepackage{colortbl}
%\usepackage{makecell}
\usepackage{graphicx}
\usepackage[c2]{optidef}
\usepackage{placeins}
\usepackage{booktabs, longtable,subcaption}
\usepackage[flushleft]{threeparttable}
\usepackage{tabularray}
\usepackage[singlelinecheck=false ]{caption}
\usepackage{etoolbox}
\usepackage{pdflscape}
\usepackage[automark]{scrlayer-scrpage}
%\usepackage{amsmath}

% ---------------------------------------------------

% Settings --------------------------------

\crefformat{equation}{equation~#2#1#3}
\crefformat{equation}{equation~#2#1#3}
\Crefformat{equation}{Equation~#2#1#3}
\crefformat{figure}{figure~#2#1#3}
\Crefformat{figure}{Figure~#2#1#3}
\crefformat{table}{table~#2#1#3}
\Crefformat{table}{Table~#2#1#3}
\Crefrangeformat{table}{Tables~#3#1#4~to~#5#2#6}
\Crefrangeformat{figure}{Figures~#3#1#4~to~#5#2#6}

\Crefmultiformat{table}{Table~#2#1#3}%
{ and~#2#1#3}% the second argument is what comes after the first reference
{, #2#1#3}% for three or more, separator format
{ and~#2#1#3}% for the final reference in a series

\Crefmultiformat{figure}{Figure~#2#1#3}%
{ and~#2#1#3}% the second argument is what comes after the first reference
{, #2#1#3}% for three or more, separator format
{ and~#2#1#3}% for the final reference in a series

%\renewcommand{\theequation}{\roman{equation}}
%\setkomafont{disposition}{\bfseries}
%\usepackage[noblocks]{authblk}

%\renewcommand*{\Authsep}{, }
%\renewcommand*{\Authand}{, }
%\renewcommand*{\Authands}{, }
%\renewcommand\Affilfont{\small}
%\setkomafont{title}{\normalfont\bfseries}
%\makeatletter
%\patchcmd{\@maketitle}{\titlefont\small}{\titlefont\tiny}{}{}
%\makeatother
%\renewcommand{\abstractwidth}{\textwidth}



%\usepackage{fancyhdr}
%\pagestyle{fancy}
%\fancyhead{}
%\fancyhead[RO,LE]{\textbf{Consumer Sentiments}}


\clearpairofpagestyles
%\ohead{\myshorttitle} %running header on right
%\ihead{\myshorttitle} %running header on left
\cfoot*{\pagemark} % Centered page number at the footer

\makeatletter
\@ifpackageloaded{caption}{}{\usepackage{caption}}
\AtBeginDocument{%
\ifdefined\contentsname
  \renewcommand*\contentsname{Table of contents}
\else
  \newcommand\contentsname{Table of contents}
\fi
\ifdefined\listfigurename
  \renewcommand*\listfigurename{List of Figures}
\else
  \newcommand\listfigurename{List of Figures}
\fi
\ifdefined\listtablename
  \renewcommand*\listtablename{List of Tables}
\else
  \newcommand\listtablename{List of Tables}
\fi
\ifdefined\figurename
  \renewcommand*\figurename{Figure}
\else
  \newcommand\figurename{Figure}
\fi
\ifdefined\tablename
  \renewcommand*\tablename{Table}
\else
  \newcommand\tablename{Table}
\fi
}
\@ifpackageloaded{float}{}{\usepackage{float}}
\floatstyle{ruled}
\@ifundefined{c@chapter}{\newfloat{codelisting}{h}{lop}}{\newfloat{codelisting}{h}{lop}[chapter]}
\floatname{codelisting}{Listing}
\newcommand*\listoflistings{\listof{codelisting}{List of Listings}}
\makeatother
\makeatletter
\makeatother
\makeatletter
\@ifpackageloaded{caption}{}{\usepackage{caption}}
\@ifpackageloaded{subcaption}{}{\usepackage{subcaption}}
\makeatother
\usepackage{bookmark}
\IfFileExists{xurl.sty}{\usepackage{xurl}}{} % add URL line breaks if available
\urlstyle{same}
\hypersetup{
  pdftitle={some title},
  hidelinks,
  pdfcreator={LaTeX via pandoc}}


\title{some title}

\vspace{-2em}
\date{}
\begin{document}
\maketitle
\vspace{-3.5em}
\thispagestyle{empty}




\newpage
\pagenumbering{arabic}  
\setcounter{page}{1} 

\section{Introduction}\label{introduction}

Lorem ipsum dolor sit amet, consectetur adipiscing elit. Sed non risus.
Suspendisse lectus tortor, dignissim sit amet, adipiscing nec, ultricies
sed, dolor. Cras elementum ultrices diam. Maecenas ligula massa, varius
a, semper congue, euismod non, mi.

Proin porttitor, orci nec nonummy molestie, enim est eleifend mi, non
fermentum diam nisl sit amet erat. Duis semper. Duis arcu massa,
scelerisque vitae, consequat in, pretium a, enim. Pellentesque congue.
Ut in risus volutpat libero pharetra tempor.

Cras vestibulum bibendum augue. Praesent egestas leo in pede. Praesent
blandit odio eu enim. Pellentesque sed dui ut augue blandit sodales.
Vestibulum ante ipsum primis in faucibus orci luctus et ultrices posuere
cubilia Curae; Aliquam nibh. Mauris ac mauris sed pede pellentesque
fermentum.

some papers include Acemoglu and Scott
(\citeproc{ref-acemoglou1994}{1994}) and also
(\citeproc{ref-angrist_effect_1992}{Angrist and Krueger 1992};
\citeproc{ref-carroll_does_1994}{Carroll, Fuhrer, and Wilcox 1994}). We
also have refered to Priya and Sharma
(\citeproc{ref-priya_sharma2024}{2024})

\section{Tables}\label{tables}

We collect data of household sentiments from Consumer Pyramid Household
Survey (CPHS)\footnote{The Consumer Pyramid Household Survey (CPHS) is
  conducted thrice every year since 2014 by the Centre for Monitoring
  Indian Economy (CMIE). Under CPHS, a large panel of sample Indian
  households are surveyed. This large longitudinal dataset is widely
  acknowledged as representative of the Indian economy. For details see;
  \url{https://consumerpyramidsdx.cmie.com/}}. It is a large
longitudinal data set, representative of Indian economy. CPHS collects
data of household sentiments of India since April, 2016. To collect the
sentiment data, a generic Indian household h is surveyed thrice in a
year, e.g.; a household surveyed in April, 2016 is surveyed again in
August, 2016 by CPHS for the collection of the sentiments data and so
on. To assess the sentiments, CPHS asks questions about the present
conditions as well as the future expectations of the household financial
position, and the business condition. In the process, to assess the
present conditions, CPHS asks the following 2 questions to the
households - (I) Compared to a year ago, how is your family faring
financially these days?; and (II) Do you think that this is generally a
good or bad times to buy things like furniture, refrigerator,
television, two-wheeler, and car? Along with this, CPHS asks the
following 3 questions to assess the short-run and the long-run future
expectations of the households - (III) How do you think that a year from
now, financially, your family would be faring?; (IV) How would you
describe the financial and business conditions in our country in the
next 12 months?; and (V) What do you think would be the financial and
business conditions in our country in the next 5 years? The answer to
questions (I), and (III) are recorded as Better, Same and Worse, and
accordingly a numerical value, 1, 0, -1 is assigned. On the other hand,
answers to questions (II), (IV) and (V) are recorded as Good time,
uncertain time and Bad time, and accordingly a numerical value, 1, 0 and
-1 is assigned to the answer.

Along with sentiments, we also collect data of household's monthly
expenditure on 8 major food groups, and fuel \& lighting from CPHS from
April 2016. The 8 major food groups include - (1) cereals; (2) oils and
fats; (3) fruits; (4) pulses and products; and (5) milk and milk
products; (6) meat, fish and egg; (7) vegetables and spices; and (8)
sweets and snacks. Along with the above mentioned 8 food groups, we also
collect data on household expenditure share for food \& fuel. We find
that the 8 food groups and the fuel and lightning contribute almost 92\%
of the expenditure for the Indian households. {\Cref{tbl-desc}} reports
the descriptive statistics of the data collected from CPHS\^{}{[}The
lower panel of {\Cref{tbl-desc}} shows the impact of Covid-19 on
household sentiments. The significant rise in the proportion of
pessimistic households, and the corresponding decline in the proportion
of optimistic households as depicted in the lower panel of
{\Cref{tbl-desc}} shows the negative impact of the Covid-19 on the
psyche of Indian households. By using the difference of the optimistic
households and the pessimistic households, we calculate the balance
statistics to give a graphical representation of the pessimistic impact
of Covid-19 on the psyche of Indian households. For details, see and the
discussion in Using this data, and by using the methodology described
below, we calculate two types of expenditure minimizing consumption
bundles for the Indian households -- (i) food bundle: consisting of the
8 food groups mentioned above; and (ii) food \& fuel bundle: consisting
of the fuel \& lighting along with the 8 food groups mentioned above.

\begin{table}

\caption{\label{tbl-desc}Descriptive Statistics}

\centering{

    \centering
     % Overall table caption
    

    \small % Reduce font size

    % Subtable 1 caption
    \caption*{(a): Demographic Variables} 

    % First Subtable: Table 1(a)
    \begin{subtable}{\textwidth}
    \label{tbl-des-a}
        \centering
        \renewcommand{\arraystretch}{0.9}
        \begin{tabular}{p{5cm} c}
            \toprule
            \textbf{Variable} & \textbf{Mean/Proportion} \\
            \midrule
            Income & 19,834.12 \\
            Age & 46.34 \\
            \addlinespace
            \textbf{Education} & \\
            \quad Less than 5 & 27.5 \\
            \quad 5-10 & 56.4 \\
            \quad 10-12 & 8.7 \\
            \quad 13-15 & 7.0 \\
            \quad 15+ & 0.4 \\
            \addlinespace
            \textbf{Gender} & \\
            \quad Male & 88 \\
            \quad Female & 12 \\
            \addlinespace
            \textbf{Marital Status} & \\
            \quad Married & 85 \\
            \quad Unmarried & 15 \\
            \addlinespace
            \textbf{Geographic Location} & \\
            \quad Rural & 25 \\
            \quad Urban & 75 \\
            \addlinespace
            \textbf{Occupation} & \\
            \quad Agriculture and Allied & 15.4 \\
            \quad Manufacturing, Industry and Auto & 34.1 \\
            \quad Services, Media, Health & 50.3 \\
            \quad Others & 0.2 \\
            \bottomrule
        \end{tabular}
    \end{subtable}

    \vspace{1.5em} % Add some vertical space between subtables

    % Subtable 2 caption
    \caption*{(b): Sentiment Variables} 

    % Second Subtable: Table 1(b)
    \begin{subtable}{\textwidth}
    \label{tbl-des_b}
        \centering
        \renewcommand{\arraystretch}{0.9}
        \begin{tabular}{p{3cm} p{2cm} c c c c c c}
            \toprule
            \textbf{Variable} & \textbf{Response} & \multicolumn{2}{c}{\textbf{Full Sample}} & \multicolumn{2}{c}{\textbf{Pre-Covid Periods}} & \multicolumn{2}{c}{\textbf{Covid Periods}} \\
            \cmidrule(lr){3-4} \cmidrule(lr){5-6} \cmidrule(lr){7-8}
            & & \textbf{N} & \textbf{\%} & \textbf{N} & \textbf{\%} & \textbf{N} & \textbf{\%} \\
            \midrule
            QFP & Bad & 12,610 & 21.4 & 5,097 & 12.9 & 7,513 & 38.8 \\
                & Same & 34,245 & 58.2 & 24,010 & 60.8 & 10,235 & 52.9 \\
                & Good & 12,016 & 20.4 & 10,407 & 26.3 & 1,609 & 8.3 \\
            \addlinespace
            QBC & Bad & 13,671 & 23.2 & 5,544 & 14.0 & 8,127 & 42.0 \\
                & Same & 30,628 & 52.0 & 20,894 & 52.9 & 9,734 & 50.3 \\
                & Good & 14,572 & 24.8 & 13,076 & 33.1 & 1,496 & 7.7 \\
            \midrule
            \multicolumn{2}{l}{Total N} & 58,871 & & 39,514 & & 19,357 & \\
            \bottomrule
        \end{tabular}
    \end{subtable}

}

\end{table}%

\FloatBarrier

Along with this, we also collect data on the price index of the
aforementioned eight food groups, as well as fuel and lighting, from
MoSPI\footnote{See; \href{https://mospi.gov.in/}{mospi.gov.in} for the
  data of price index.}. MoSPI directly reports the price index for the
first five food groups---(1) to (5) listed above. However, it separately
provides the price index for the following food items:(i) Meat and fish
(ii) Egg (iii) Vegetables (iv) Spices (v) Sweets (vi) Snacks. Using the
price indices of these food items and their corresponding weights, we
calculate the monthly price index for the remaining three food
groups:(6) Meat, fish, and eggs (7) Vegetables and spices (8) Sweets and
snacks

\phantomsection\label{refs}
\begin{CSLReferences}{1}{0}
\bibitem[\citeproctext]{ref-acemoglou1994}
Acemoglu, Daron, and Andrew Scott. 1994. {``Consumer Confidence and
Rational Expectations: Are Agents' Beliefs Consistent with the
Theory?''} \emph{The Economic Journal} 104 (422): 1--19.
\url{http://www.jstor.org/stable/2234671}.

\bibitem[\citeproctext]{ref-angrist_effect_1992}
Angrist, Joshua D., and Alan B. Krueger. 1992. {``The {Effect} of {Age}
at {School} {Entry} on {Educational} {Attainment}: {An} {Application} of
{Instrumental} {Variables} with {Moments} from {Two} {Samples}.''}
\emph{Journal of the American Statistical Association} 87 (418):
328--36. \url{https://doi.org/10.1080/01621459.1992.10475212}.

\bibitem[\citeproctext]{ref-carroll_does_1994}
Carroll, Christopher D., Jeffrey C. Fuhrer, and David W. Wilcox. 1994.
{``Does {Consumer} {Sentiment} {Forecast} {Household} {Spending}? {If}
{So}, {Why}?''} \emph{The American Economic Review} 84 (5): 1397--1408.
\url{http://www.jstor.org/stable/2117779}.

\bibitem[\citeproctext]{ref-priya_sharma2024}
Priya, Pragati, and Chandan Sharma. 2024. {``On Transmission Channels of
Energy Prices and Monetary Policy Shocks to Household Consumption:
Evidence from India.''} \emph{Energy Economics} 136: 107723.
https://doi.org/\url{https://doi.org/10.1016/j.eneco.2024.107723}.

\end{CSLReferences}




\end{document}
